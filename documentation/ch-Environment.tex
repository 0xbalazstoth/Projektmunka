\chapter{Környezet és annak kialakítása}

\section{WSL telepítése}
\begin{flushleft}
    \textbf{A Docker szoftverhez szükséges a WSL telepítése.}
\end{flushleft}
\begin{enumerate}
    \item WSL engedélyezése PowerShell-ben
    \begin{listing}[H]
        \begin{minted}[linenos, numbersep=-10pt, breaklines]{powershell}
            dism.exe /online /enable-feature /featurename:Microsoft-Windows-Subsystem-Linux /all /norestart
        \end{minted}
        \caption{WSL engedélyezése}
        \label{code:wsl_ena}
    \end{listing}
    \item Virtual Machine Platform engedélyezése
    \begin{listing}[H]
        \begin{minted}[linenos, numbersep=-10pt, breaklines]{powershell}
            dism.exe /online /enable-feature /featurename:VirtualMachinePlatform /all /norestart
        \end{minted}
        \caption{VMP engedélyezése}
        \label{code:vmp}
    \end{listing}
    \item Ubuntu telepítése WSL segítségével
    \item \begin{listing}[H]
        \begin{minted}[linenos, numbersep=-10pt, breaklines]{powershell}
            wsl --install -d Ubuntu
        \end{minted}
        \caption{Ubuntu WSL}
        \label{code:ubuntu_wsl}
    \end{listing}
\end{enumerate}