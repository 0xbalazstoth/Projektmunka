\graphicspath{ {./assets/} }

\chapter{DNS szerver}

\section{Indítás}
Ahhoz, hogy használni tudjuk az email klienst, kritikus fontosságú a DNS szervernek a futtatása. Természetesen arra vonatkozik ez a megkötés, ha lokálisan futtatjuk az email szervert!
\begin{itemize}
    \item \verb|docker-compose up -d|
\end{itemize}

\begin{flushleft}
    Parancs futtatása után látható, hogy sikeresen elindult a DNS szerver.
    \begin{center}
        \includegraphics[width=0.9\textwidth]{docker-up-dns-server.png}
    \end{center}
\end{flushleft}

\section{Konfiguráció}
\begin{flushleft}
    Magát a konfigurációt webes környezetben hajthatjuk végre, amely látható is, hogy a \textbf{5380} porton fut. Cím, amelyen a konfigurációt elvégezhetjük:
    \begin{itemize}
        \item \verb|http://localhost:5380/|
    \end{itemize}
\end{flushleft}

\subsection{Zóna hozzáadása}
Megadott paraméterek:
\begin{itemize}
    \item Zóna neve: \verb|oemail.io|
    \item Típus: \verb|Primary Zone|
\end{itemize}

\subsection{Record hozzáadása}