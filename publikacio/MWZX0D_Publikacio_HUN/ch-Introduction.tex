\section{Bevezetés}
A digitális kommunikáció mára már szinte elengedhetetlenné vált az emberek mindennapi életében. Sajnos ezzel párhuzamosan a spam üzenetek terjedése is exponenciálisan növekszik, ezzel kihívást állítva a rendszerekre, amelyek a nemkívánatos tartalmakat szűrik. Kritikus fontosságú, hogy az adott szoftver amit használunk az üzenetek küldésére és fogadására, az megbízhatóan szűrje azokat. \\

A tanulmány azt mutatja be, hogy hogyan alkalmazható a PyTorch keretrendszer és a természetes nyelvi feldolgozási megközelítések összehangoltan egy intelligens spam szűrő rendszer kialakítására. Segítségével a bemeneti adatok alapján képes megtanulni, hogy az általunk megadott adatok alapján melyek vannak besorolva általános vagy gyanús üzenetként.