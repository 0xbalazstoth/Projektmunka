\section{Introduction}
Digital communication has become almost indispensable in people's daily lives.
Unfortunately, spam is growing exponentially at the same time, challenging the systems that filter out unwanted content.
It is critical that the software we use to send and receive message filters them reliably. \\

This paper shows how the PyTorch framework and natural language processing approaches can be used together to design an intelligent spam filtering system.
It is able to recognise if the given data is general or suspicious message.

\subsection{Typical patterns in email spam}
\begin{itemize}
    \item{Phishing emails are designed to impersonate a trusted organisation and lure recipients into revealing sensitive information such as usernames, passwords or any valuable data.}
    \item{Malware emails send attachments or links to malicious software designed to infect the recipient's device with viruses, ransomware or other malicious programs.}
    \item{The advance payment scam, these emails promise large sums of money in exchange for a small advance.}
    \item{Fake lottery or prize scams, which falsely claim that recipients have won a lottery prize and often ask for personal details or payment to claim the alleged prize.}
    \item{Survey emails designed to collect personal data for fraudulent purposes.}
\end{itemize}

These are the most common types of email spam but the list could be endless.