\section{Why PyTorch?}
In 2016, Facebook's AI research group, now known as Meta, took the lead in creating the PyTorch framework and generously shared it with the global community as an open-source resource.
PyTorch has gained recognition for its outstanding qualities, being praised for its exceptional simplicity, impressive flexibility, and inherent efficiency.
These remarkable features have solidified PyTorch's position as a fundamental and highly regarded tool in the fields of artificial intelligence and machine learning.

\begin{table}[htp]
    \begin{center}
    \caption{Comparing PyTorch with Keras}
    \label{tab1}
    \small
        \begin{tabular}{| c | c | c |}
            \hline
            Category & PyTorch & Keras \\
            \hline
            API Level & Low & High \\
            \hline
            Datasets & Large datasets, & Smaller \\
             & high-performance & datasets \\
            \hline
            Debugging & Good debugging & Challenging \\
             & capabilities & \\
            \hline
            Pretrained models & Yes & Yes \\
            \hline
            % Speed & Fast, high-performance, & Slow, low-performance \\
            Speed & Fast, high- & Slow, low- \\
             & performance & performance \\
            \hline
            Written in & Lua, & Python \\
            \hline
            Visualization & Limited & Depends on \\
            & & backend \\
            \hline
        \end{tabular}
    \end{center}
\end{table}

Table 1 provides a detailed comparison between the PyTorch and Keras frameworks. [1]
The key factor influencing the choice of PyTorch is its impressive performance and the ability to handle large datasets seamlessly.
This pivotal decision is grounded in the framework's robust capabilities, making it a reliable choice for our study.