\chapter{Környezet és annak kialakítása}

\section{WSL telepítése}
\begin{flushleft}
    \textbf{A Docker szoftverhez szükséges a WSL telepítése.}
\end{flushleft}
\begin{enumerate}
    \item WSL engedélyezése PowerShell-ben
    \begin{listing}[H]
        \begin{minted}[linenos, numbersep=-10pt, breaklines]{powershell}
            dism.exe /online /enable-feature /featurename:Microsoft-Windows-Subsystem-Linux /all /norestart
        \end{minted}
        \caption{WSL engedélyezése}
        \label{code:wsl_ena}
    \end{listing}
    \item Virtual Machine Platform engedélyezése
    \begin{listing}[H]
        \begin{minted}[linenos, numbersep=-10pt, breaklines]{powershell}
            dism.exe /online /enable-feature /featurename:VirtualMachinePlatform /all /norestart
        \end{minted}
        \caption{VMP engedélyezése}
        \label{code:vmp}
    \end{listing}
    \item Ubuntu telepítése WSL segítségével
    \item \begin{listing}[H]
        \begin{minted}[linenos, numbersep=-10pt, breaklines]{powershell}
            wsl --install -d Ubuntu
        \end{minted}
        \caption{Ubuntu WSL}
        \label{code:ubuntu_wsl}
    \end{listing}
\end{enumerate}

\section{GIT telepítése}
\begin{flushleft}
    GIT hivatalos oldala, \textbf{\href{https://git-scm.com/downloads}{https://git-scm.com/downloads}}, amit szükséges telepíteni.
\end{flushleft}

\subsection{Távoli repository klónozása}
\begin{flushleft}
    A projekt klónozása a GIT verziókezelő rendszer segítségével történik, amely lehetővé teszi a teljes forráskód, konfigurációs fájlok és egyéb szükséges erőforrások másolását a távoli repository-ból a helyi gépre. A klónozási folyamat első lépése a GIT telepítése, majd ezt követően GIT kliens alkalmazás vagy parancssorban a célmappában adjuk ki a \verb|git clone| parancsot.
\end{flushleft}
\begin{listing}[H]
    \begin{minted}[linenos, numbersep=-10pt, breaklines]{bash}
        git clone https://github.com/0xbalazstoth/Projektmunka.git
    \end{minted}
    \caption{Távoli repository klónozása}
    \label{code:repo_clone}
\end{listing}

\subsection{MongoDBCompass telepítése}
\begin{flushleft}
    MongoDB hivatalos oldala \textbf{\href{https://www.mongodb.com/try/download/compass}{https://www.mongodb.com/try/download/compass}}, telepítése opcionális.
\end{flushleft}

\subsection{.env fájlok beállítása}
\begin{flushleft}
    \verb|.env| fájlokban állíthatóak be a projekt olyan paraméterei, mint például
    \begin{itemize}
        \item Adatbázis kapcsolat
        \item Hoszt IP-címe
        \item SMTP, IMAP szerver adatok
    \end{itemize}
\end{flushleft}